\documentclass[11pt]{exam}

\usepackage{umut}
\usepackage{mathptmx}
\usepackage{uprog}

\printanswers

\pagestyle{headandfoot}
	\lhead{Cogs 502 -- Prog. \& Log. \\ Fall 2017}
	\chead{Quiz 3}
\rhead{Nov 28}
\lfoot{}
\pointname{\%}


\begin{document}
\qformat{\bf Question \thequestion%
\ifthenelse{\equal{\thepoints}{}}{}{\quad (\thepoints)} \hfill}

\makebox[\textwidth]{Name of the Student:\enspace\hrulefill}
\vspace{10pt}
\begin{center}
\fbox{\parbox{6in}{\bf\centering 2 questions in 30 minutes}}
\end{center}
\vspace{10pt}

\begin{questions}

\question[50] Write a {\bf function} that removes any repeated elements in a given list and returns the resulting list. For example, \pyv{[1,2,1,3,2,1]} should yield \pyv{[1,2,3]}.

\fillwithlines{\stretch{1}}

\newpage

\question[50] Write a {\bf function} that takes two arguments: a list of integers \pyv{k} and an integer \pyv{n}. Your function should return \pyv{True} if it is IMPOSSIBLE to find in \pyv{k} two consecutive integers whose multiplication is greater than \pyv{n}, and \pyv{False} otherwise. Items \pyv{x} and \pyv{y} are consecutive, if the index of \pyv{y} minus the index of \pyv{x} is 1.

\fillwithlines{\stretch{1}}
 
\end{questions}
\end{document}
