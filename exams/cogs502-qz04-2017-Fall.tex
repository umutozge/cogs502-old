\documentclass[11pt]{exam}

\usepackage{umut}
\usepackage{mathptmx}
\usepackage{uprog}

\printanswers

\pagestyle{headandfoot}
	\lhead{Cogs 502 -- Prog. \& Log. \\ Fall 2017}
	\chead{Quiz 4}
\rhead{Dec 12}
\lfoot{}
\pointname{\%}


\begin{document}
\qformat{\bf Question \thequestion%
\ifthenelse{\equal{\thepoints}{}}{}{\quad (\thepoints)} \hfill}

\makebox[\textwidth]{Name of the Student:\enspace\hrulefill}
\vspace{10pt}
\begin{center}
\fbox{\parbox{6in}{\bf\centering 2 questions in 30 minutes}}
\end{center}
\vspace{10pt}

\begin{questions}

\question[60] 
Take any list of \pyv{0}'s and \pyv{1}'s, e.g.\ \pyv{[0,1,1,0,1,1,0,1,0,1]}; process the list from left to right, flipping every element whose right and left neighbors are identical. Flipping an item means making it \pyv{0} if it's \pyv{1}, and making it \pyv{1} if it's \pyv{0}. The example list should yield \pyv{[0,1,1,1,0,0,0,0,0,1]}.

\fillwithlines{\stretch{1}}

\newpage

\question[40] What is the output of the following program:   

\begin{ucodeframe}
\inputpygments{python}{code/functioncall.py}
\end{ucodeframe}


\fillwithlines{\stretch{1}}
 
\end{questions}
\end{document}
